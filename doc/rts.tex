\documentclass{article}

\title{Overview of the Compiler Run-Time System}
\author{Iavor S. Diatchki}

\usepackage{graphicx}

\begin{document}

We aim for a fairly standard run-time system, that supports garbage
collection and efficient tail calls.  There are two memory regions used
by the run-time system: the stack, and the heap.  The stack is used
to store function arguments and local variables.  The heap stores
dynamically allocated objects.

\section{Calling Convention}
\label{call-sec}

The compiler uses the `standard' calling convention: arguments are 
allocated on the the stack by the caller, and deallocated by the callee.
This convention enables the implementation of efficient tail-calls,
because functions are not required to return to deallocate their arguments.
Figure \ref{frame-fig} shows a typical stack frame.
\begin{figure}[!ht]
\includegraphics{frame.ps}
\caption{A stack frame}
\label{frame-fig}
\end{figure}
The stack grows towards smaller addresses.  Notice that the return address
of a function is placed on the stack before the arguments (i.e., it is the
first argument to every function).  This means that when we make a tail call
to a function with a different number of arguments we do not need to adjust
the return address.  On the negative side, it also means that we cannot
use the {\tt call} instruction to call a function, instead we use an 
explicit {\tt jmp}.  The code we generate for a function call is like this:
\begin{verbatim}
    push $L
    push A1
    push A2
    jmp F
    .long scav
L:  ... 
\end{verbatim}
The instruction \verb|.long scav| provides data for the garbage collector
and is described in Section \ref{gc-sec}.




\section{Garbage Collection}
\label{gc-sec}

There is another reason that we cannot use the {\tt call} instruction to 
call functions, and it has to do with the way we arrange for garbage collection
support.  When we start a garbage collection, we need to know an initial
set of life pointers to objects in the heap (the so called `root set').
The pointers that are on the run-time stack and any global pointer variables
form a conservative approximation to this set, because they are variables 
in functions that will be resumed at some point.  
So we need to know which values on the stack are pointers in
the heap and which are not.   While generating code the compiler can keep
track of the locations of the pointers in a single stack frame.  However,
what frames are present on the stack cannot be determined until run time.
To work around this problem, we observe that the return address for a function
uniquely identifies the frame of the calleer.  So we arrange that the location
before the return address of a function contains a pointer to the scavange
method for the frame.  Thus, to `walk the stack' we need to know how to
scavange the first frame, and then we can use the return addresses to  
scavange the following frames.  We also need to know when we have
finished processing the stack.  We do this by ensuring that the code
that calls the {\tt main} function of a program places a 0 before the code
where execution continues.  Thus, walking the stack terminates when the
scavange method for a frame is 0:
\begin{verbatim}
void walk_stack(DWord* frame, ScavFun scav) {
  do {
    DWord* ret;
    frame = scav(frame);
    ret   = (DWord*)(*frame);
    scav  = (ScavFun)ret[-1];
    --frame;
  } while (scav != 0);
  scav_globals();
}
\end{verbatim}



\end{document}



